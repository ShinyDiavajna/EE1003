\let\negmedspace\undefined
\let\negthickspace\undefined
\documentclass[journal]{IEEEtran}
\usepackage[a5paper, margin=10mm, onecolumn]{geometry}
%\usepackage{lmodern} % Ensure lmodern is loaded for pdflatex
\usepackage{tfrupee} % Include tfrupee package

\setlength{\headheight}{1cm} % Set the height of the header box
\setlength{\headsep}{0mm}     % Set the distance between the header box and the top of the text

\usepackage{gvv-book}
\usepackage{gvv}
\usepackage{cite}
\usepackage{amsmath,amssymb,amsfonts,amsthm}
\usepackage{algorithmic}
\usepackage{graphicx}
\usepackage{textcomp}
\usepackage{xcolor}
\usepackage{txfonts}
\usepackage{listings}
\usepackage{enumitem}
\usepackage{mathtools}
\usepackage{gensymb}
\usepackage{comment}
\usepackage[breaklinks=true]{hyperref}
\usepackage{tkz-euclide} 
\usepackage{listings}
% \usepackage{gvv}                                        
\def\inputGnumericTable{}                                 
\usepackage[latin1]{inputenc}                                
\usepackage{color}                                            
\usepackage{array}                                            
\usepackage{longtable}                                       
\usepackage{calc}                                             
\usepackage{multirow}                                         
\usepackage{hhline}                                           
\usepackage{ifthen}                                           
\usepackage{lscape}
\begin{document}

\bibliographystyle{IEEEtran}
\vspace{3cm}

\title{11.16.2.2.7}
\author{EE24BTECH11058 - P.Shiny Diavajna}
% \maketitle
% \newpage
% \bigskip
{\let\newpage\relax\maketitle}

\renewcommand{\thefigure}{\theenumi}
\renewcommand{\thetable}{\theenumi}
\setlength{\intextsep}{10pt} % Space between text and floats


\numberwithin{equation}{enumi}
\numberwithin{figure}{enumi}
\renewcommand{\thetable}{\theenumi}
\textbf{Question:}\\
A die is thrown\\
D: The event that a number less than 4 appears.\\
E: The event that an even number greater than 4 appears.\\
Find the probability of the event $D-E$.\\


\textbf{Solution:}\\
Probabilities of the given events 
\begin{align}
         P\brak{D}&=\frac{1}{2}\\
         P\brak{E}&=\frac{1}{6}\\
\end{align}  
It can be observed that, D and E are disjoint events . Therefore, 
\begin{align}
     P\brak{DE}&=0
\end{align}

We,know that 
\begin{align}
        P\brak{D-E}&= P\brak{D}-P\brak{DE}\\
        P\brak{D-E}&= \frac{1}{2} - 0\\
        P\brak{D-E}&= \frac{1}{2}
\end{align}


The code simulates rolling a fair six-sided die $10^5$ times and calculates the probabilities of three events:
\begin{itemize}
    \item D(rolling a number less than 4)
    \item E(rolling an even number greater than 4)
    \item D-E(numbers in D but not in E)
\end{itemize}
It then visualizes these probabilities in the below plot.         

\newpage
\textbf{Plot:}\\
\begin{figure}[ht]
   \centering
   \includegraphics[width=\columnwidth]{figs/fig.png}
\end{figure}

\end{document}
